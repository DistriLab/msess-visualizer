\documentclass[acmsmall,10pt,review,anonymous]{acmart}\settopmatter{printfolios=true}

\usepackage{booktabs}
\usepackage{amsmath}
\usepackage{amssymb}
%\usepackage{color}
% \usepackage[usenames,dvipsnames]{color}
\usepackage{xcolor}
\setcounter{tocdepth}{3}
\usepackage{graphicx}
\usepackage{caption}
\usepackage{listings}
\usepackage{url}
\usepackage[linesnumbered,lined,sright,ruled,titlenotnumbered]{algorithm2e}
\usepackage{algpseudocode}
\usepackage{enumerate}
% \usepackage{enumitem}
\usepackage[shortlabels]{enumitem}
\usepackage{fancyvrb}
\usepackage{frameit}
\usepackage{graphics}
\usepackage{latexsym}
\usepackage{multirow}
\usepackage{proof}
\usepackage{sidecap}
\usepackage{url}
\usepackage{wrapfig}
\usepackage{flushend}
\usepackage{ebproof}
\usepackage{hyperref}
% \usepackage{bm}
\usepackage{verbatim}
\usepackage{listings}
\usepackage[utf8x]{inputenc}
% \usepackage{amsmath}
\usepackage{verbatim}
% \usepackage{moreverb}
\usepackage{listings}
% \usepackage{calligra}
% \usepackage{mathtools}
\usepackage{subfig}
\usepackage{euscript}

\usepackage[T1]{fontenc}

\DeclareSymbolFont{stmry}{U}{stmry}{m}{n}
\DeclareMathSymbol\mergemap\mathbin{stmry}{"21}

\DeclareSymbolFont{ugrf@m}{U}{eur}{m}{n}
\SetSymbolFont{ugrf@m}{bold}{U}{eur}{b}{n}
\DeclareMathSymbol{\Upupsilon}{\mathord}{ugrf@m}{"07}

% \newcommand{\mergemap}{M}
\newcommand{\cL}{{\cal L}}




\setcopyright{none}
\pagenumbering{arabic}
% DOI
\acmDOI{}

% ISBN
\acmISBN{}

% Conference
\acmConference[]{}{}{}
\acmYear{}
\copyrightyear{}

\acmPrice{}

%\settopmatter{printacmref=false} % Removes citation information below abstract
\renewcommand\footnotetextcopyrightpermission[1]{} % removes footnote with conference information in first column
%\pagestyle{plain} % removes running headers

\begin{document}
\captionsetup[figure]{labelfont=bf,textfont=normalfont,singlelinecheck=on}
\setlist[itemize]{leftmargin=*}
\setlist[enumerate]{leftmargin=*}
% \special{papersize=8.5in,11in}
% \setlength{\pdfpageheight}{\paperheight}
% \setlength{\pdfpagewidth}{\paperwidth}


\title{A Visualizer and Simulator for Session Protocols}


\author{}
\affiliation{}
\definecolor{darkgreen}{HTML}{008000}

\newcommand{\cb}{{\textit{communicates-before}}}
\newcommand{\chan}{\mathcal{C}}
\newcommand{\lchan}{\tilde{\mathcal{C}}}
\newcommand\commonspec[1]{\code{Common(#1)}}
\newcommand\partyspec[2]{\code{Party(#1,#2)}}
\newcommand\changlob[2]{\overline{\chan}(#1,#2)}
\newcommand\gloabspec[3]{\code{Glob(#1,#2,#3)}}
% \newcommand\chanspec[3]{\code{Channel{\langle}#1,#2,#3{\rangle}}}
%\newcommand{\perchanprj}[3]{\code{{#1}{\prj}{#2}{\langle}{#3}{\rangle}}}
\newcommand{\perchanprj}[3]{\code{{#1}{\prj}{#2}{\prj}{#3}}}
\newcommand{\perpartyprj}[2]{\code{{#1}{\prj}{#2}}}
\newcommand{\All}{\code{All}}
\newcommand{\codeS}[1]{\ensuremath{\it #1}}
\newcommand{\prj}{\mytt{\#}}
\newcommand{\chpred}[2]{{\cal C}(#1,#2)}
\newcommand{\wnsay}[1]{{{\color{blue}({\bf WN:} \textit{#1})}}}
% \newcommand{\wnsay}[1]{}
\newcommand{\wnnay}[1]{}
\newcommand{\asay}[1]{{{\color{darkgreen}{\bf A:} \textit{#1}}}}
% \newcommand{\asay}[1]{}
\newcommand{\anay}[1]{}
%% \newcommand{\scsay}[1]{{{\color{cyan}({\bf SC:} \textit{#1})}}}
\newcommand{\scsay}[1]{}
\newcommand{\scnay}[1]{}
\newcommand{\doubt}[1]{{\color{red}{#1}}}
\newcommand{\betw}[1]{\ensuremath{{\cdots}#1{\cdots}}}
\def\defeq{\ensuremath{\,\triangleq}}
\def\cd{\,{\cdot}\,}
\def\compl{{\sim}}
\def\pst{\omega}
\def\st{\ensuremath{\sigma}}
\def\sta{\ensuremath{\sigma'}}
\def\stap{\ensuremath{\sigma''}}
\def\St{\ensuremath{\mathrm{\Delta}}}
\def\Sta{\ensuremath{\mathrm{\Delta'}}}

\def\mw{\ensuremath{-\hspace{-1ex}\circledast~}}

\def\Q{\ensuremath{\code{Q}}}
\def\P{\ensuremath{\code{P}}}
\def\Pow{\ensuremath{\mathcal{P}}}
\def\B{\ensuremath{\mathsf{B}}}
\def\S{\ensuremath{\mathcal{S}}}
\def\SH{\ensuremath{\mathsf{SH}}}
\def\PSH{\ensuremath{\Pow_\SH}}

%\newcommand{\rearr}[2]{\ensuremath{\mathsf{rearr}(#1)(#2)}}
\newcommand{\filter}[2]{\ensuremath{\mathsf{filter}(#1)(#2)}}
%\newcommand{\exec}[2]{\ensuremath{\mathsf{exec}(#1)(#2)}}
%\newcommand{\abs}[1]{\ensuremath{\mathsf{abs}(#1)}}

\def\AllSpec{\ensuremath{\mathsf{AllSpec}}}
\def\T{\ensuremath{\mathcal{T}}}
\def\Pt{\ensuremath{\mathcal{P}^\top}}

\def\rearr{\ensuremath{\mathsf{unfold}}}
\def\exec{\ensuremath{\mathsf{exec}}}
\def\abs{\ensuremath{\mathsf{abs}}}
\def\absa{\ensuremath{\mathsf{abs_a}}}

\def\Rearr{\ensuremath{\mathsf{Unfold}}}
\def\Exec{\ensuremath{\mathsf{Exec}}}
\def\Abs{\ensuremath{\mathsf{Abs}}}

\newcommand{\Pre}{\ensuremath{\mathsf{Pre}}}
\newcommand{\Post}{\ensuremath{\mathsf{Post}}}
\newcommand{\Curr}{\ensuremath{\mathsf{Curr}}}

\def\svar{\ensuremath{\mathsf{svars}}}
\def\rec{\ensuremath{\mathsf{rec}}}
\def\Pairs{\ensuremath{\mathsf{Pairs}}}
\def\Spec{\ensuremath{\mathsf{Spec}}}
\def\mspec{\ensuremath{\myit{mspec}}}
\newcommand{\fv}[1]{\ensuremath{\mathsf{fv}(#1)}}
\newcommand{\lv}[1]{\ensuremath{\mathsf{lv}(#1)}}
\def\Refine{\ensuremath{\mathsf{Refine}}}
\def\ReachVar{\ensuremath{\mathsf{ReachVar}}}
\def\b{\ensuremath{\mathsf{b}}}
\newcommand{\old}[1]{\ensuremath{\mathsf{old}(#1)}}
\newcommand{\ol}[1]{\ensuremath{#1^*}}
\newcommand{\lift}[1]{\ensuremath{#1^\dag}}

\newcommand{\mtt}[1]{\ensuremath{\mathtt{#1}}}
\newcommand{\mbf}[1]{\ensuremath{\mathbf{#1}}}
\newcommand{\msf}[1]{\ensuremath{\mathsf{#1}}}
\newcommand{\sevent}[2]{\ensuremath{#1^{#2}_S}}
\newcommand{\revent}[2]{\ensuremath{#1^{#2}_R}}
\newcommand{\event}[2]{\ensuremath{#1^{(#2)}}}
% \def\ev{\ensuremath{\varepsilon}}
\def\ev{\ensuremath{E}}
\newcommand{\eventf}[3]{\ensuremath{#1^{#2}{{\at}#3}}}
\newcommand{\transf}[3]{\ensuremath{#1{{\at}#3}}}

%\newcommand{\pre}[1]{\ensuremath{\mathsf{Pre}_\mtt{#1}}}
%\newcommand{\post}[1]{\ensuremath{\mathsf{Post}_\mtt{#1}}}

\def\Var{\ensuremath{\mathsf{Var}}}
\def\LVar{\ensuremath{\mathsf{LVar}}}
\def\SVar{\ensuremath{\mathsf{SVar}}}

\def\nul{\texttt{null}}

\newcommand{\denoleft}{\ensuremath{|\hspace{-.06cm}[}}
\newcommand{\denoright}{\ensuremath{]\hspace{-.06cm}|}}
\newcommand{\semantic}[3]{{\denoleft}#1{\denoright}_{#2}#3}
\newcommand{\semantica}[3]{{\denoleft}#1{\denoright}_{#2}^\msf{A}#3}
\newcommand{\semantici}[3]{{\denoleft}#1{\denoright}_{#2}#3}
\newcommand{\triple}[3]{\{ #1 \}~ {\color{blue}{#2}} ~\{ #3 \}}

\def\vl{\ensuremath{\mathsf{val}}}
\def\rf{\ensuremath{\mathsf{ref}}}
\def\dispose{\ensuremath{\mathsf{dispose}}}
\def\new{\ensuremath{\mathsf{new}}}
\def\cskip{\ensuremath{\mathsf{skip}}}
\newcommand{\cif}[3]{\ensuremath{\mathsf{if}~#1~\mathsf{then}~#2~\mathsf{else}~#3~\mathsf{fi}}}
\newcommand{\cwhile}[2]{\ensuremath{\mathsf{while}~#1~\mathsf{do}~#2~\mathsf{od}}}
\def\known{\ensuremath{\mathsf{known}}}
\def\unknown{\ensuremath{\myit{unk}}}
\def\local{\ensuremath{\mathsf{local}}}
\def\return{\ensuremath{\mathsf{return}}}

\def\D{\Delta}
\def\Pst{\Upsilon}
\def\S{\mathsf{S}}
\def\H{\mathsf{H}}
\def\Re{\mathsf{R}}
\def\subst{\rho}
\newcommand{\hyphen}[0]{\ensuremath{\mbox{-}}}
\newcommand{\pair}[2]{\ensuremath{(#1, #2)}}%{\ensuremath{\langle #1;~ #2 \rangle}}
\newcommand{\unk}[2]{\ensuremath{\myit{unk}_{#1}{(#2)}}}

\def\TODO{\textsf{TODO}~}

%%%%%%

\def\State{\ensuremath{\varsigma}}
\def\mem{\code{heap}}
\def\stk{\code{stk}}
\def\so{\textit{sizeof}}
\def\sso{\textit{ssizeof}}
\newcommand{\hp}{\ensuremath{\mu}}
\newcommand{\sk}{\ensuremath{\sigma}}
\newcommand{\err}{\ensuremath{\myit{er}}}
\newcommand{\errdead}{\ensuremath{\bot_{\it deadlock}}}
\newcommand{\memerr}{\ensuremath{\myit{er}_1}}
\newcommand{\othererr}{\ensuremath{\myit{er}_2}}
%\newcommand{\incsk}[1]{\ensuremath{\stk{:=}\stk{+}\myit{sizeof}(#1)}}
%\newcommand{\inchp}[1]{\ensuremath{\mem{:=}\mem{+}\myit{sizeof}(#1)}}
\newcommand{\incsk}[1]{\ensuremath{\myit{inc\_stk}(#1)}}
\newcommand{\inchp}[1]{\ensuremath{\myit{inc\_hp}(#1)}}
\newcommand{\decsk}[1]{\ensuremath{\myit{dec\_stk}(#1)}}
\newcommand{\dechp}[1]{\ensuremath{\myit{dec\_hp}(#1)}}
\newcommand{\cimplies}{\approx\!\!>}
%%%%%%%%%%%%%%%%
%\newcommand{\report}[1]{ }
%\newcommand{\paper}[1]{#1}
\newcommand{\acm}[1]{}
%%%%%%%%%%%%%%%%
\newcommand{\hide}[1]{}
\newcommand{\nil}{\btt{null}}
\newcommand{\res}{\btt{res}}
\newcommand{\ra}{\,\rightarrow\,}
\newcommand{\emp}{\btt{emp}}
\newcommand{\veq}{\ensuremath{\equiv}}
\newcommand{\self}{\btt{root}}
%\newcommand{\root}{\btt{root}}
\newcommand{\wwith}{\btt{inv}}
\newcommand{\equivimp}{\ensuremath{\Longleftrightarrow}}
\newcommand{\revimp}{\ensuremath{\Longleftarrow}}
\newcommand{\ndet}{\ensuremath{\wedge}}
\newcommand{\puret}{\ensuremath{\pi}}
%\newcommand{\simp}{\ensuremath{\textrm{--}\!\btt{*}}}
\newcommand{\simp}{{\bf ;}}
\newcommand{\minv}{\ensuremath{\iota}}
\newcommand{\ptr}{\ensuremath{\gamma}}
\newcommand{\nm}{\ensuremath{\phi}}
\newcommand{\heap}{\ensuremath{\kappa}}
\newcommand{\constr}{\ensuremath{\mathrm{\Phi}}}
\newcommand{\pconstr}{\ensuremath{\phi}}
\newcommand{\sat}[1]{\myit{SAT}(#1)}
\newcommand{\hmodels}[2]{\ensuremath{#1\,{\models}\,#2}}
\newcommand{\abduct}[2]{\ensuremath{#1 \rhd #2}}
\newcommand{\nabduct}[2]{\ensuremath{#1 \ntriangleright #2}}
\newcommand{\entail}[2]{\ensuremath{#1 \vdash #2}}
\newcommand{\entailA}[2]{\ensuremath{#1 \triangleright #2}}
\newcommand{\nentail}[2]{\ensuremath{#1 \nvdash #2}}
\newcommand{\entailD}[3]{\ensuremath{#1{\vdash}#2\,{\sep}\,#3}}
\newcommand{\entailH}[3]{\ensuremath{#1{\vdash}#2\,{\sep}\,#3}}
%\newcommand{\entailE}[4]{\ensuremath{#2{\vdash}#3\,{\sep}\,#4}}
\newcommand{\entailE}[4]{\ensuremath{#2{\vdash}_{#1}#3\,{\sep}\,#4}}
\newcommand{\entailV}[3]{\entailE{V}{#1}{#2}{#3}}
\newcommand{\entailK}[5]{\ensuremath{#3{\vdash}^{#1}_{#2}#4\,{\sep}\,#5}}
\newcommand{\entailVV}[3]{\entailK{\heap}{V}{#1}{#2}{#3}}
\newcommand{\entailI}[3]{\ensuremath{#2 \vdash_{#1} #3}}
\newcommand{\myit}[1]{\textit{#1}}
\newcommand{\mytt}[1]{\texttt{#1}}
% \newcommand{\U}{\ensuremath{{\btt{U}}}}
\newcommand{\R}{\ensuremath{{\btt{R}}}}
\newcommand{\Le}{\ensuremath{{\btt{L}}}}
\newcommand{\Sh}{\ensuremath{{\btt{S}}}}
\newcommand{\A}{{\myit{A}}}
\newcommand{\at}{\mytt{@}}
\newcommand{\isview}{\myit{IsView}}
\newcommand{\isdata}{\myit{IsData}}
\newcommand{\hc}[2]{\ensuremath{{#1}{::}#2}}
\newcommand{\hx}[1]{\hc{\code{x}}{#1}}
\newcommand{\mtype}[2]{\ensuremath{{#1}{ \ :}#2}}
\def\primeV{\myit{prime}}
\def\entword{ENT}
\newcommand{\rulen}[1]{\ensuremath{{\bf \scriptstyle #1}}}
\newcommand{\entrule}[1]{\ensuremath{\scriptstyle \entword-#1}}
\newcommand{\entrulen}[1]{[\underline{{\bf \scriptstyle \entword-}\rulen{#1}}]}
\newcommand{\oprulen}[1]{[\underline{{\bf \scriptstyle \opword-}\rulen{#1}}]}
\def\opword{OP}
\newcommand{\lemrulen}[1]{[{{\bf \scriptstyle LEM-}\rulen{#1}}]}
\newcommand{\lemnormrulen}[1]{\lemrulen{NORM-#1}}
\newcommand{\ebpentail}[2]{
	\begin{prooftree}
		\Hypo{\entrulen{#1}}
		\Infer[no rule]1{{\begin{prooftree}[rule style=simple] #2 \end{prooftree}}}
	\end{prooftree}
}
\newcommand{\ebpsem}[2]{
	\begin{prooftree}
		\Hypo{\oprulen{#1}}
		\Infer[no rule]1{{\begin{prooftree}[rule style=simple] #2 \end{prooftree}}}
	\end{prooftree}
}
\newcommand{\entwlbl}[2]{{\begin{array}{c} { \entrulen{#1} } \\ {#2} \end{array} }}
%\newcommand{\entwlbl}[2]{{ { \entrulen{#1} } \\ {#2} }}
\newcommand{\verirulen}[1]{[\underline{{\bf \scriptstyle FV-}\rulen{#1}}]}
\newcommand{\reasonrulen}[1]{[\underline{{\bf \scriptstyle FR-}\rulen{#1}}]}
\newcommand{\markred}[1]{{{\color{red}#1}}}
\newcommand{\markblue}[1]{{{\color{blue}#1}}}
% \newcommand{\ebpverif}[2]{
% 	\begin{prooftree}
% 		%\Hypo{\verirulen{#1}}
% 		\Hypo{[\underline{{\bf \scriptstyle }\rulen{#1}}]}
% 		\Infer[no rule]1{{\begin{prooftree}[rule style=simple] #2 \end{prooftree}}}
% 	\end{prooftree}
%       }
\newcommand{\ebpverif}[2]{
  \ifthenelse{\equal{#1}{} }{
    \begin{prooftree}[rule style=simple] #2 \end{prooftree}
  }{
    \begin{prooftree}
      % \Hypo{\verirulen{#1}}
      \Hypo{[\underline{{\bf \scriptstyle }\rulen{#1}}]}
      \Infer[no rule]1{{\begin{prooftree}[rule style=simple] #2 \end{prooftree}}}
    \end{prooftree}
  }}

%
%\newcommand{\entail}[2]{\ensuremath{#1 \vdash #2}}
%\newcommand{\ebpverif}[2]{
	%\begin{prooftree}
		%\Hypo{\verirulen{#1}}
		%\Infer[no rule]1{ #2 }
	%\end{prooftree}
%}
%\newcommand{\hform}[3]{\ensuremath{{#1}{::}{#2}{\langle}{#3}{\rangle}}}
\newcommand{\hformn}[3]{\ensuremath{{\btt{#1}}{\mapsto}{\btt{#2}}{\langle}{\btt{#3}}{\rangle}}}
%\newcommand{\view}[2]{\ensuremath{#1{\langle}{#2}{\rangle}}}
%\newcommand{\hform}[3]{\ensuremath{{#2}{\langle}{#1,#3}{\rangle}}}
%\newcommand{\mapview}[3]{\ensuremath{{#1}{\mapsto}{#2}{\langle}{#3}{\rangle}}}
\newcommand{\view}[2]{\ensuremath{#1{(}{#2}{)}}}
\newcommand{\pview}[2]{\ensuremath{\overline{#1}{(}{#2}{)}}}
\newcommand{\sspec}[3]{\ensuremath{\view{\mathcal{#1}}{#2,#3}}}
\newcommand{\hform}[3]{\ensuremath{{#2}({#1,#3})}}
\newcommand{\uhform}[3]{\ensuremath{\code{H}({#2})({#1,#3})}}
\newcommand{\mapview}[3]{\ensuremath{{#1}{\mapsto}{#2}{(}{#3}{)}}}
\newcommand{\transit}{\rightarrow}
\newcommand{\arrimp}{\ensremath{\,\btt{*}\!\!\!\rightarrow}\,}
% \newcommand{\state}[1]{\ensuremath{{\langle}#1{\rangle}}}
% \newcommand{\drule}[2]{\ensuremath{\frac{#1}{#2}}}
% \newcommand{\drulea}[2]{\ensuremath{\begin{array}{c}{#1}\\\hline{#2}\end{array}}}
% \newcommand{\dyn}[2]{\ensuremath{\state{#1}{\hookrightarrow}\state{#2}}}
% \newcommand{\dyns}[2]{\ensuremath{\state{#1}{\hookrightarrow^*}\state{#2}}}
% \newcommand{\dynf}[1]{\ensuremath{\state{#1}{\not\hookrightarrow^*}}}
% \newcommand{\dynerr}[1]{\ensuremath{\state{#1}{\hookrightarrow}{\err}}}
% \newcommand{\dynmemerr}[1]{\ensuremath{\state{#1}{\hookrightarrow}{\memerr}}}
% \newcommand{\dynothererr}[1]{\ensuremath{\state{#1}{\hookrightarrow}{\othererr}}}
\def\postst{\myit{Post}}


%used in heap constraint solving
\def\rewrite{\leadsto}
%\def\map{\myit{map}}
\def\map{\myit{X{\!}Pure}}
\def\mappair{\myit{X{\!}Pure}_P}
%\def\mapview{\ensuremath{\myit{X{\!}Pure}_V}}
\def\mapself{\myit{mapself}}
\def\mapi{\myit{mapI}}
%\def\mapviewi{\myit{mapvI}}
\def\mapselfi{\myit{mapsI}}
\def\findu{\myit{find}}
\def\fold{\myit{fold}_V}
\def\foldN{\myit{fold}}
\def\foldK{\myit{fold}^{\heap}}
\newcommand{\foldKI}[1]{\myit{fold}^{#1}}
\def\unfold{\myit{unfold}}
\def\alias{\myit{alias}}
\def\merge{\myit{merge}}
\def\overlap{\myit{overlap}}
\def\ex{{\bf ex}}

%def\M{\myit{M}}
%def\C{\myit{C}}
%def\I{\myit{I}}
%def\E{\myit{E}}
%def\A{\myit{A}}
%def\B{{\myit{B}}}
\def\D{\Delta}
%def\S{{\cal S}}
%def\H{{\cal H}}
%def\W{{\cal M}}
%\def\loop{\ensuremath{\alpha}}
\def\myloop{\ensuremath{\alpha}}
\def\arity{\myit{arity}}
\def\max{\myit{max}}
\def\fresh{\myit{fresh}}
\def\where{\btt{where}}
\def\bag{\ensuremath{\code{B}}}
\def\bagunion{\sqcup}
\def\bagintersect{\sqcap}
\def\bagsubsume {\sqsubset}
\def\bagsubtract{-}
\def\zerobag{\{{\bf 0}\}}
\def\zerobslash{\emp}
\newcommand{\load}[1]{\mytt{load}{\langle}#1{\rangle}}
\newcommand{\store}[1]{\mytt{store}{\langle}#1{\rangle}}
%\def\store{\mytt{store}}
\newcommand{\const}[1]{\mytt{const}{\langle}#1{\rangle}}
%\def\iconst{\mytt{iconst}}
\def\invoke{\mytt{invoke}}
\def\invokeTail{\mytt{invoke}_{Tail}}
\def\new{\mytt{new}}
\def\dispose{\mytt{dispose}}
\def\bool{\btt{bool}}
\def\int{\btt{int}}
\def\longint{\mytt{long}}
\def\float{\mytt{float}}
\def\double{\mytt{double}}
\def\fixpt{\myit{fixpt}}
\def\rec{\mytt{Rec}}
\def\void{\btt{void}}
\def\type{\myit{type}}
\def\reference{\mytt{ref}}
\def\object{\mytt{object}}
\def\objectset{{\rm\bf ObjectType}}
\def\mn{\myit{m}}
\def\F{{\cal F}}%stack frame size
\def\phiinv{\phi_{\myit{inv}}}
\def\phirec{\phi_{\myit{rec}}}
\def\pre{\constr_{\myit{pr}}}
\def\post{\constr_{\myit{po}}}
\def\T{\Gamma}%type for the stack frame
\def\eqt{\myit{eq}_\tau}
%\def\stk{\pi} %current stack frame
\def\FS{\Pi}  %the stack of frames
\def\true{\btt{true}}
\def\false{\btt{false}}
%\def\a{\alpha}
\def\a{s}
\def\restrict{\myit{enrich}}%context enhance
\newcommand{\parg}[1]{\ensuremath{{\langle}#1{\rangle}}}
%% \newcommand{\hash}{\ensuremath{{\scriptstyle \#}}}
\newcommand{\num}[1]{|#1|}
\newcommand{\ifEq}[2]{\mytt{if}~#1~#2}
\newcommand{\whileEq}[1]{\mytt{while}~#1}
\newcommand{\myif}[3]{\btt{if}~(#1)~#2~\btt{else}~#3}
\newcommand{\mywhile}[3]{\btt{while}~#1~\btt{where}~#3~\btt{do}~#2}
\newcommand{\mybind}[3]{\mytt{bind}~#1~\mytt{to}~#2~\mytt{in}~#3}
\def\skipcmd{\mytt{nop}}
\newcommand{\defs}{\ensuremath{=_{df}}}
\newcommand{\snd}[1]{\myit{snd}(#1)}
\newcommand{\guarding}{{\ensuremath{\rightarrow}}}
\newcommand{\nonneg}[1]{\myit{notneg}(#1)}
\newcommand{\code}[1]{{\ensuremath{\tt #1}}}
\newcommand{\sm}[1]{\mbox{$#1$}}
\newcommand{\btt}[1]{{\tt #1}}
\newcommand{\todo}[1]{{\bf(* #1 *)}}
\newcommand{\passref}{\btt{ref}}
\newcommand{\ensures}{\btt{ensures}}
\newcommand{\requires}{\btt{requires}}

\newcommand{\llin}[1]{\texttt{#1}}

%%below renamed for llncs format
\newtheorem{thm}{Theorem}
%\newtheorem{lemma}{Lemma}
\newtheorem{pty}{Property}
%\newtheorem{cor}[thm]{Corollary}
\newtheorem{defn}{Definition}
\newtheorem{idfn}{Informal Definition}
\def\proof{\vspace{-0.2cm}\noindent{\bf Proof}\hspace*{0.5cm}}
%\newcommand{\tverb}[1]{\mbox{\verb|#1|}}

%\newcommand{\tcv}[2]{#1\,\vdash\,#2}
\newcommand{\tcv}[3]{#1\,\vdash\,#2,\,#3}
\newcommand{\tcvT}[2]{\tcv{}{#1}{#2}}


%frame bound inference
\newcommand{\fsr}[5]{#1 \vdash_{\small\myit{F}} #2 \rightsquigarrow #3, #4, #5}
\newcommand{\fsinf}[3]{\ensuremath{\frac{\begin{array}{c}[\underline{\rulen{FS-#1}}]\\[0.5ex]
#2\end{array}}{#3}}}
%\newcommand{\rulen}[1]{\ensuremath{{\bf \scriptstyle #1}}}

%abstract state inference
\newcommand{\asr}[4]{#1 \vdash_{\small\myit{A}} #2 \rightsquigarrow #3, #4}
\newcommand{\asinf}[3]{\ensuremath{\frac{\begin{array}{c}[\underline{\rulen{AS-#1}}]\\[0.5ex]
#2\end{array}}{#3}}}

%stack space inference
\newcommand{\ssrA}[2]{\ssr{\myit{a}}{#1}{#2}}
\newcommand{\ssr}[3]{#1 \vdash_{\small\myit{S}} #2 \rightsquigarrow #3}
\newcommand{\ssinf}[3]{\ensuremath{\frac{\begin{array}{c}[\underline{\rulen{SS-#1}}]\\[0.5ex]
#2\end{array}}{#3}}}

%heap space inference
\newcommand{\hsrX}[5]{#5,#1 \vdash_{\myit{H}} #2 \rightsquigarrow #3,#4}
\newcommand{\hsr}[4]{\hsrX{#1}{#2}{#3}{#4}{\myit{a}}}
%\newcommand{\hsr}[4]{\myit{a},#1 \vdash_{\myit{H}} #2 \rightsquigarrow #3,#4}
\newcommand{\hsinf}[3]{\ensuremath{\frac{\begin{array}{c}[\underline{\rulen{HS-#1}}]\\[0.5ex]
#2\end{array}}{#3}}}

%op semantics
\newcommand{\oprerr}[5]{\ensuremath{\config{#1}{#2}{#3}{#4}\hookrightarrow #5}}
%\newcommand{\config}[4]{\ensuremath{{\langle}{#1},{#2},{#3},{#4}{\rangle}}}
%\newcommand{\opr}[8]{\ensuremath{\config{#1}{#2}{#3}{#4} \hookrightarrow \config{#5}{#6}{#7}{#8}}}
%\newcommand{\oprd}[4]{\opr{\FS}{#1}{\O}{#2}{\FS}{#3}{\O}{#4}}
\newcommand{\config}[4]{\ensuremath{{\langle}{#1},{#2},{#3},{#4}{\rangle}}}
\newcommand{\opr}[8]{\ensuremath{\config{#1}{#2}{#3}{#4} \hookrightarrow \config{#5}{#6}{#7}{#8}}}
\newcommand{\oprd}[4]{\opr{#1:\FS}{\O}{#2}{\ss,\mem}{#3:\FS}{\O}{#4}{\ss,\mem}}
\newcommand{\oprds}[6]{\opr{#1:\FS}{\O}{#2}{\ss,\mem}{#3:\FS}{#4}{#6}{\ss,#5}}
\newcommand{\oprc}[5]{\config{#1}{#2}{#3}{#4}\hookrightarrow^* {#5}}
\newcommand{\oprule}[3]{\ensuremath{\frac{\begin{array}{c}[\underline{\rulen{OP-#1}}]\\[0.5ex]
#2\end{array}}{#3}}}

\def\EI{\epsilon}
\def\Y{\Upsilon}
\def\L{\Lambda}
\def\O{\omega}
\def\i{\iota}
%\def\mem{\mytt{h}}%available heap space
%\def\ss{\mytt{s}}%available stack space
\def\MemErr{\mytt{MemInadequate}}
\def\OF{\mytt{FrameErr}}
\def\cf{\myit{f}}
\def\ret{\mytt{ret}}
\def\size{\myit{size}}
\def\cxt{\mytt{cxt}}
\newcommand\reqs[4]{\ensuremath{#1 \vdash #2 \rightsquigarrow #3,#4 }}

\def\G{\Gamma}
%\def\res{\myit{res}}
%\def\nochange{\myit{no}\!{\cal X}}
\def\nochange{\myit{nochange}}
\def\FSV{\myit{V}}
\newcommand{\sep}{\ensuremath{*}}
\newcommand{\sepc}{\ensuremath{*}}
\newcommand{\rview}[2]{\code{\ensuremath{#1{(}{\self,#2}{)}}}}
\newcommand{\sepimpl}{\ensuremath{\code{-\!\!\!\!-\!\!\!*}}}
\newcommand\eq[3]{\ensuremath{\myit{eq}_{#1}(#2,#3)}}
\newcommand{\tj}[5]{#1,#2\,{\vdash}\,#3~{::}~#4, #5}
%\newcommand{\tr}[3]{\ensuremath{\frac{\begin{array}{c}[\underline{\rulen{S-#1}}]\\[0.5ex]
%#2\end{array}}{#3}}}
\newcommand{\tr}[3]{\ensuremath{\frac{\begin{array}{c}[\underline{\rulen{#1}}]\\[0.5ex]
#2\end{array}}{#3}}}
\newcommand{\hlr}[3]{\ensuremath{\frac{\begin{array}{c}\reasonrulen{#1}\\[0.5ex]%[\underline{\rulen{#1}}]\\[0.5ex]
#2\end{array}}{#3}}}
\newcommand{\hlrs}[3]{\ensuremath{\frac{\begin{array}{c}%\verirulen{#1}\\[0.5ex]%[\underline{\rulen{#1}}]\\[0.5ex]
#2\end{array}}{#3}}}
\newcommand{\htriple}[3]{\ensuremath{\vdash \{#1\}\,#2\,\{#3\}}}
%\newcommand{\rulen}[1]{\ensuremath{{\bf \scriptstyle #1}}}
\newcommand{\iview}[2]{\ensuremath{#1{\langle}#2{\rangle}}} %use it rather than tt
\newcommand{\ihc}[2]{\ensuremath{#1{::}#2}}                 %use it rather than tt
\newcommand{\subtr}[3]{\entailH{#1}{#2}{#3}}
%\newcommand{\subtr}[3]{\ensuremath{#1\,{\vdash}\,#2,#3}}
\newcommand{\reasonrulenm}[1]{[\underline{{\bf \scriptstyle }\rulen{#1}}]}
\newcommand{\hlrm}[3]{\ensuremath{\frac{\begin{array}{c}\reasonrulenm{#1}\\[0.5ex]%[\underline{\rulen{#1}}]\\[0.5ex]
#2\end{array}}{#3}}}

%static semantics
\newcommand{\raws}[9]{{#1} ; {#2} ; {#3} ;{#4} \vdash {#5} :: {#6} , {#7},{#8} ,{#9}}
\newcommand{\rawss}[5]{\T ; \D; \L; \Y  \vdash {#1} :: {#2} , {#3}, {#4}, {#5}}

%from sound-cav.tex
\newcommand{\mymodel}{\models}
\def\Flds{\myit{Fields}}
\def\Store{\myit{Heaps}}
\def\Stack{\myit{Stacks}}
\def\Locations{\myit{Loc}}
\def\Val{\myit{Val}}
\def\Var{\myit{Var}}
\newcommand{\mo}[3]{#1,#2 \mymodel #3}
\newcommand{\notmo}[3]{#1,#2 \not\mymodel #3}
\newcommand{\mos}[1]{\ensuremath{\mo{s}{h}{#1}}}
\newcommand{\mosp}[1]{\ensuremath{{s}\mymodel{#1}}}
\def\iffs{\mbox{\tt iff~}}
\def\dom{\myit{dom}}
\def\loc{\iota}
\newcommand{\s}[1]{[\![#1]\!]s}
\def\inv{\myit{inv}}
\def\getP{\myit{getProperty}}
\newcommand{\menv}[2]{\frac{\begin{array}{c}#1\end{array}}{#2}}
\def\fresh{\myit{fresh}}
\def\length{\myit{length}}

\newcommand{\mysf}[1]{\textsf{\bf\small #1}}
% \newcommand{\Top}{\mbox{$\top$}}
\newcommand{\presize}{\phi_{\myit{pr}}}
\newcommand{\postsize}{\phi_{\myit{po}}}
\newcommand{\sizev}{{\cal V}}
\newcommand{\pure}{\ensuremath{\pi}}
\newcommand{\sarg}[1]{{\la}#1{\ra}}
\def\la{\langle}
\def\ra{\rangle}
%\def\bag{B}
\def\bagconstr{\varphi}

\newcommand{\pnt}[2]{\ensuremath{#1 {\mapsto} #2}}
\newcommand{\lseg}[2]{\ensuremath{\mathsf{ls}(#1, #2)}}
\newcommand{\lsegs}[3]{\ensuremath{\mathsf{ls}(#1, #2, #3)}}


\newcommand{\hipsleek}{\textsc{Hip/Sleek}~}
\newcommand{\hip}{\textsc{Hip}~}
\newcommand{\sleek}{\textsc{Sleek}~}
\newcommand{\M}{\mathsf{M}}

\newcommand{\absr}[3]{\ensuremath{\frac{\displaystyle{#1}}
    {\displaystyle{#2}}{\textsc{#3}}}}

\newcommand{\join}[2]{\ensuremath{\msf{join}(#1,#2)}}
\newcommand{\pjoin}[2]{\ensuremath{\msf{join_\pure}(#1,#2)}}
\newcommand{\ljoin}[2]{\ensuremath{\lift{\msf{join}}(#1,#2)}}
\newcommand{\widen}[2]{\ensuremath{\msf{widen}(#1,#2)}}
\newcommand{\pwiden}[2]{\ensuremath{\msf{widen_\pure}(#1,#2)}}
\newcommand{\lwiden}[2]{\ensuremath{\lift{\msf{widen}}(#1,#2)}}
\newcommand{\joining}{\ensuremath{\msf{join}}}
\newcommand{\widening}{\ensuremath{\msf{widen}}}
%\newcommand{\abs}{\leadsto}
\newcommand{\cfm}{\Lambda}
\newcommand{\junk}{\textit{junk}}

%%%%%%

\def\hip{\textsc{Hip}~}
\def\sleek{\textsc{Sleek}~}
\def\Prog{\ensuremath{\mathcal{P}}}
\def\shps{\ensuremath{\mathcal{S}}}
\def\fail{\textsf{fail}}
\def\FAIL{\textsf{FAIL}}
\def\entfailed{\myit{entfail}}
\newcommand{\numerical}[1]{\ensuremath{\mathsf{pureV}(#1)}}
\newcommand{\abstraction}[1]{\ensuremath{\mathsf{gen\_abs}(#1)}}
\newcommand{\postprime}[1]{\ensuremath{\mathsf{prime}(#1)}}
\newcommand{\Reach}[2]{\ensuremath{\mathsf{Reach}(#1, \{#2\})}}
\newcommand{\Local}[2]{\ensuremath{\mathsf{Local}(#1, \{#2\})}}
\newcommand{\Frame}[2]{\ensuremath{\mathsf{Frame}(#1, \{#2\})}}
\newcommand{\pr}[1]{\ensuremath{\mathsf{ex\_quan}(#1)}}
\newcommand{\pureres}[1]{\code{#1}}
\def\Sp{\Prog}
\newcommand{\lbl}[1]{\ensuremath{\myit{lbl}(#1)}}

%%%%%%%%%%%%%%%%%%%%%%%%%%%%%%%%%%%%%%%%%%%%%%%%%%
%%%%%%%%%%%%%%%%%%%%%%%%%%%%%%%%%%%%%%%%%%%%%%%%%%
\lstset{ language=[ANSI]C,
  basicstyle=\ttfamily\small,
  keywordstyle=\color{black}\ttfamily,
  % keywordstyle=\color{blue},
  % numbers=left,
  % numberstyle=\scriptsize,
  % stepnumber=1,
  % numbersep=5pt,
  commentstyle=\small\color{red},
  % backgroundcolor=\color[rgb]{0.9,1.0,1.0},
  % frame=single,
  framexleftmargin=0mm,
  xleftmargin=0mm,
  xrightmargin=0mm,
  % resetmargins=true,
  % rulesepcolor=\color{red!20!green!20!blue!20!},
  captionpos=b,
  showspaces=false,
  showstringspaces=false,
  showtabs=false,
  tabsize=4,
  breaklines=true,
  % lineskip=0pt;
  extendedchars=false,
  % morekeywords={self,requires,ensures,data},
  escapeinside= {(*@} {@*)}
}
%%%%%%%%%%%%%%%%%%%%%%%%%%%%%%%%%%%%%%%%%%%%%%%%%%
%%%%%%%%%%%%%%%%%%%%%%%%%%%%%%%%%%%%%%%%%%%%%%%%%%



%%%%%%%%%%%%%%%%%%%%%%%%%%%%%%%%%%%%%%%%%%%%%%%%%%
%%%%%%%%%%%%%%%%%%%%%%%%%%%%%%%%%%%%%%%%%%%%%%%%%%


%space def

\newcommand{\dist}{0em}
\newcommand{\distv}{0em}
\newcommand{\disth}{0em}
\newcommand{\distt}{0em}

\newcommand{\distf}{-1.6ex}


%prop definition
%\theoremstyle{definition}
%\newtheorem{law}{Law}
%\theoremstyle{plain}
%\newtheorem{prop}[law]{Property}

%%% Local Variables:
%%% mode: latex
%%% TeX-master: "main"
%%% End:


\newcommand{\sstate}[1]{\textcolor{MidnightBlue}{\code{#1}}}
\newcommand{\obar}[1]{\code{\bar{#1}}}

\newcommand{\viewbody}{\L}
% \newcommand{\prot}{\myit{G}}
\newcommand{\prot}{{G}}
\newcommand{\cproj}{{Z}}
\newcommand{\cprojtail}{{Z^t}}
\newcommand{\cprojres}{{Z^\prime}}
\newcommand{\gp}{H}
\newcommand{\prott}{\ensuremath{\prot^\mathcal{T}}}
\newcommand{\proj}{{L}}
% \newcommand{\projp}{\proj^p}
\newcommand{\projp}{\Upupsilon}
\newcommand{\projc}{\proj}
\newcommand{\unkbox}{\square}
\newcommand{\sessend}{end}

\newcommand{\send}[2]{
  \ifthenelse{\equal{#1}{} }{
     !{#2}
   }{  !{#1}{\cd}{#2}}
 }
\newcommand{\recv}[2]{
   \ifthenelse{\equal{#1}{} }{
     ?{#2}
   }{  ?{#1}{\cd}{#2}}
 }
\newcommand{\sendc}[3]{
  \ifthenelse{\equal{#2}{} }{
     #1!{#3}
  }{ #1!{#2}{\cd}{#3} }
}
\newcommand{\recvc}[3]{
  \ifthenelse{\equal{#2}{} }{
     #1?{#3}
  }{ #1?{#2}{\cd}{#3} }
}

\newcommand{\trans}[2]{\unkbox {#1}{\cd}{#2}}

\newcommand{\seq}{;} %sequence
%\newcommand{\useq}{*} %order insensitive conj
% \newcommand{\useq}{\circledast}
\newcommand{\useq}{*}
\newcommand{\disj}{\vee}
\newcommand{\chani}[3]{\ensuremath{\view{\chan}{#1,#2,#3}}}
\newcommand{\lchani}[3]{\ensuremath{\view{\lchan}{#1,#2,#3}}}
%second order variable
%second order variable for heap formula
%second order variable for session
\newcommand{\hov}{V}
\newcommand{\hovars}{\hov}
\newcommand{\hovarh}{\hov}
\newcommand{\hosets}{\mathcal{I}}
\newcommand{\hoseth}{\mathcal{I}}
\newcommand{\vdashs}{\vdash_{s}}
\newcommand{\vdashe}{\vdash_{e}}
\newcommand{\entailS}[3]{\ensuremath{#1 \vdash #2 ~\leadsto ~#3}}

%%%%%%%%%%%%%%%%%%%%%%%%%%%%%%%%%%%%%
% SYNC related stuff (fence,cdl, etc)
%%%%%%%%%%%%%%%%%%%%%%%%%%%%%%%%%%%%%
\def\gfencelit{f}
\def\genfencelit{F}
\def\pfencelit{{f}}
\def\sync{\ensuremath{\zeta}}
\def\await{await}
\def\down{down}
\def\cons{}
\def\acc{+}

%\newcommand{\fence}[3]{
%\ifthenelse{\equal{#1}{-} }{
  %{#3}^\bullet({#2})
%}{
  %\ifthenelse{\equal{#1}{+} }{
		%{#3}^\circ({#2})
%}{
		%{#3}^\circ({#2},#1)
%}}}
\newcommand{\fence}[4]{
\ifthenelse{\equal{#1}{-} }{
  {#4}^\text{\cons}({#2},#3)
}{\ifthenelse{\equal{#1}{+} }{
{#4}^\text{\acc}({#2},#3)
}{\ifthenelse{\equal{#1}{D} }{
{\down}^\text{\acc}({#2},#3)
}{\ifthenelse{\equal{#1}{A} }{
{\await}^\text{\cons}({#2},#3)
}{{\sync}^\text{}({#2})} }
}}}

%\newcommand{\gensync}[1]{}

%%%%% global fence
\def\gfenceboth{\rightleftharpoons}
\def\gfencesend{\leftharpoonup}
\def\gfencereceive{\rightharpoondown}
\newcommand{\ggfence}[3]{\fence{#1}{#2}{#3}{\gfencelit}}
\newcommand{\gstub}[3]{\fence{#1}{#2}{#3}{\gfencelit}}
\newcommand{\gfence}[4]{
\ifthenelse{\equal{1}{#3} }{
#1{\gfencereceive}#2:#4 % fencing on the receiver
}{
\ifthenelse{\equal{2}{#3} }{
#1{\gfencesend}#2:#4   % fencing on the sender
}{
#1{\gfenceboth}#2:#4   % unknown source for fencing
}}}
\newcommand{\pfence}[3]{\fence{#1}{#2}{#3}{\pfencelit}}
\newcommand{\fences}{\mathcal{F}}
\newcommand{\stripfence}[1]{\ensuremath{#1\%\sync}}
\newcommand{\opt}[1]{[#1]}

\newcommand{\upeergfence}[2]{\ensuremath{\gfence{\unkbox}{\unkbox}{#1}{#2}}}


%%%%%%%%%%%%%%%%%%%%%%%%%%%%%%%%%%%%%
% PROJECTION related stuff
%%%%%%%%%%%%%%%%%%%%%%%%%%%%%%%%%%%%%
\newcommand{\projectsymbol}{\downharpoonright}
\newcommand{\project}[2]{(#2){\projectsymbol}_{#1}}

\newcommand{\bind}[2]{#1~$\leftarrow$~#2}
\newcommand{\mapschan}[2]{\ensuremath{#1\vDash#2}}

\def\pequiv{\dashv\vdash}
%%%%%%%%%%%%%%%%%%%%%%%%%%%%%%%%%%%%%
% MISC
%%%%%%%%%%%%%%%%%%%%%%%%%%%%%%%%%%%%%
\def\wrt{w.r.t.~}
\definecolor{projcolor}{RGB}{0, 176, 80} %green
\definecolor{protcolor}{RGB}{222, 71, 142} %pink
\definecolor{fencecolor}{RGB}{0, 176, 240} %blue

\newcommand{\setfont}[2]{{\fontfamily{#1}\selectfont #2}}

\newcommand{\savespace}{\vspace{-2mm}}
\def\dual{\ensuremath{\sim}}
\def\resid{S}
\def\chanvar{c}
\def\lchanvar{c}
\def\echanvar{\tilde{c}}
\def\rolevar{P}
\def\msgvar{\myit{m}}
\def\employ{\in}
\def\notemploy{\notin}
\def\eqdef{\mathrel{\stackrel{\makebox[0pt]{\mbox{\normalfont\tiny
          def}}}{=}}}

\def\pstate{PS}
\def\cstate{CS}
\def\tstate{TS}
\def\tconf{TC}
\def\sstore{SS}
\def\css{CSS}
\newcommand{\wait}[2]{\code{join~{#1}~{#2}}}
\def\SAFE{\btt{SAFE}}
\def\HOLE{\btt{HOLE}}
\def\ANS{\btt{ANS}}
\def\BLOCK{\btt{BLOCK}}
\def\YES{\btt{YES}}
\def\NO{\btt{NO}}
\def\safesend{\btt{S_{safe}}}
\def\saferecv{\btt{R_{safe}}}
\def\fid{\btt{FID}}

\def\idt{idt}
\def\idp{idp}
\def\idc{idc}

\newcommand\figref[1]{Fig. \ref{#1}}
\newcommand\secref[1]{Sec. \ref{#1}}
\newcommand\defref[1]{Def.\ref{#1}}

% roman god of communication
\def\sessionlogic{Mercurius}

%%%%%%%%%%%%%%%%%%%%%%%%%%%%%%%%%%%%%
% global protocol related stuff
%%%%%%%%%%%%%%%%%%%%%%%%%%%%%%%%%%%%%
\newcommand{\explicitseq}[1]{
  \ifthenelse{\equal{RR}{#1} }{
    {\downdownharpoons} %% \downdownarrows
  }{
    \ifthenelse{\equal{SS}{#1} }{
      \upupharpoons %% \upuparrows
    }{
      \ifthenelse{\equal{RS}{#1} }{
        \downupharpoons %% \downuparrows
      }{
        \ifthenelse{\equal{SR}{#1} }{
          \updownharpoons%% \updownarrows
        }{

        }
      }
    }
  }
}

\newcommand{\gseq}[1]{\ensuremath{\seq^{\explicitseq{#1}}}}
\newcommand{\gseqA}[1]{\ensuremath{\seq}}
\newcommand{\guard}[1]{\ensuremath{{\ominus}(#1)}}
\newcommand{\assume}[1]{\ensuremath{{\oplus}(#1)}}
%% \newcommand{\assumeL}[2]{\ensuremath{\oplus(#1)_{#2}}}
\newcommand{\assumeL}[2]{\ensuremath{\oplus(#1)}}

%% A->B : k <\delta>  where
%% #1 --> A
%% #2 --> B
%% #3 --> \delta
%% #4 --> k
%% #5 --> transmit ->, labelled or not
\newcommand{\transmitbase}[5]{
  \ifthenelse{\equal{#3}{} \AND \equal{#4}{}}{
    #1{#5}#2
  }{
    \ifthenelse{\NOT \equal{#3}{} \AND \equal{#4}{}}{
      #1{#5}#2:#3
    }{
      \ifthenelse{\NOT \equal{#4}{} \AND \equal{#3}{}}{
        #1{#5}#2:#4
      }{
        #1{#5}#2:#4{\langle}#3{\rangle}
      }
    }
  }
}

\newcommand{\transmit}[4]{\transmitbase{#1}{#2}{#3}{#4}{\rightarrow} }
\newcommand{\bringson}[2]{{#1}{\rightharpoonup}{#2} }
\newcommand{\delegate}[3]{\transmitbase{#1}{#2}{#3}{}{\xrightarrow{d}}}
\newcommand{\atransmit}[5]{\transmitbase{#1}{#2}{#3}{#4}{\xrightarrow{#5}}}
\newcommand{\msg}[2]{#1\,{\cdot}\,#2}
\newcommand{\omsg}[2]{#2}

%% delegation

%%%%%%%%%%%%%%%%%%%%%%%%%%%%%%%%%%%%%
% CDL related stuff
%%%%%%%%%%%%%%%%%%%%%%%%%%%%%%%%%%%%%
% global CDL
\newcommand{\gdown}[3]{#1:#2 {\langle} DOWN(#3) {\rangle}}
\newcommand{\gawait}[3]{#1:#2 {\langle} AWAIT(#3) {\rangle}}
% local CDL
\newcommand{\pdown}[1]{down(#1)}
\newcommand{\pawait}[1]{await(#1)}

% \newcommand{\gprot}[4]{
%   \ifthenelse{\equal{#4}{}}{
%     #1(#2)@I(#3) } {
%     #1(#2)@I(#3) \leadsto @O(#4)}
% }

\newcommand{\gprot}[4]{
  \ifthenelse{\equal{#4}{}}{
    #1(#2,#3) } {
    #1(#2,#3) \leadsto @O(#4)}
}


%%%%%%%%%%%%%%%%%%%%%%%%%%%%%%%%%%%%%
% causality related stuff
%%%%%%%%%%%%%%%%%%%%%%%%%%%%%%%%%%%%%
\def\atid{\textit i}
\def\atloc{\textit l}
\def\atzero{\bot}

\def\implies{\Rightarrow}

\newcommand{\addL}[1]{\ensuremath{(#1,0)}}
\newcommand{\addR}[1]{\ensuremath{(0,#1)}}
\newcommand{\addFull}[0]{\ensuremath{1}}
\newcommand{\addAt}[1]{\ensuremath{{\at}#1}}
\def\atFull{\addAt{\addFull}}
\def\atL{\addAt{\addL{\addFull}}}
\def\atR{\addAt{\addR{\addFull}}}
\def\atLno{\addL{\addFull}}
\def\atRno{\addR{\addFull}}
\def\atLL{\addAt{\addL{\atLno}}}
\def\atLR{\addAt{\addL{\atRno}}}
\def\atRL{\addAt{\addR{\atLno}}}
\def\atRR{\addAt{\addR{\atRno}}}

\newcommand{\CB}{CB}
\newcommand{\HB}{HB}
\newcommand{\SEQ}{SEQ}
\newcommand{\precx}[1]{\ensuremath{\prec_{#1}}}
\newcommand{\lt}[2]{#1\,{\prec}\,#2}
% \newcommand{\ltset}[2]{#1\,{\prec^*}\,#2}
\newcommand{\ltset}[2]{merge(#1,\,#2)}
\newcommand{\ltcb}[2]{#1\!{\prec_{\CB}}#2}
\newcommand{\lthb}[2]{#1\!{\prec_{\HB}}#2}
\newcommand{\seqhb}[2]{#1\!{\prec_{\SEQ}}#2}
\newcommand{\lthbw}[2]{#1{\preceq_{\HB}}\,#2}
\newcommand{\lthbmay}[2]{#1{\prec^{maybe}_{HB}}\,#2}
% \newcommand{\rl}[2]{#1^{(#2)}}
% same as event
\newcommand{\rl}[2]{\event{#1}{#2}}
% \newcommand{\rrl}[3]{(#1,#2)^{(#3)}}
\newcommand{\rrl}[3]{\atransmit{#1}{#2}{}{}{#3}}
\newcommand{\rk}[3]{\rl{#1}{#3}{:}{#2}}
\def\roles{\mathcal{R}}
\newcommand{\graph}[1]{
  \ifthenelse{\equal{#1}{}}{\mathcal{G}}{\mathcal{G}(#1)}}


\def\oplusseq{\otimes}
\def\oplusstar{\circledast}
\def\opluschoice{\oplus}

\def\roletyp{\mathcal{R}{{ole}}}
\def\eventtyp{\mathcal{E}{{vents}}}
\def\labeltyp{{{Nat}}}
\def\idtyp{nat}
\def\lchantyp{\mathcal{C}{{han}}}
\def\chantyp{\mathcal{{L}}{{chan}}}
\def\echantyp{\mathcal{{E}}{{ndpt}}}
%% \def\chantyp{\Var}
\def\operationtyp{\mathcal{O}}

\def\ordering{\pure}
\newcommand{\chanmap}{\ensuremath{\mathrm{\Gamma}}}
\def\rolemap{K}
\def\locof{\mathcal{L}}
\def\loccompstar{\useq}
\def\loccompseq{\seq}
\def\loccompvee{\vee}
% \def\loccompstar{\bowtie}
% \def\loccompseq{\ltimes}
% \def\loccompvee{\vee}

\def\composeseqop{h_1}
\def\composestarop{h_2}
\def\composeveeop{h_3}
\newcommand{\composeseq}[2]{{\composeseqop}(#1,#2)}
\newcommand{\composestar}[2]{{\composestarop}(#1,#2)}
\newcommand{\composevee}[2]{{\composeveeop}(#1,#2)}

\def\bordermergebase{\sqcup}
% \newcommand{\bordermerge}[1]{\bordermergebase^{#1}}
\newcommand{\bordermerge}[1]{\lfloor{#1}\rfloor}
\newcommand{\mergeseq}{\bordermerge{\loccompseq}}
\newcommand{\mergestar}{\bordermerge{\loccompstar}}
\newcommand{\mergevee}{\bordermerge{\loccompvee}}

\def\mergeset{merge\_adjacent}
\newcommand{\mergeseqtier}[2]{\mergeset(#1,#2)}
\newcommand{\mergestartier}[2]{merge\_star(#1,#2)}

% frontier
\def\mfront{F}
% backtier
\def\mback{B}
% constraints
\newcommand{\morders}{\ensuremath{\mathrm{\Pi}}}
% obligations
\def\mguards{A^{\ominus}}
\def\massum{A^{\oplus}}
\def\terminal{T}
% \def\operation{\varepsilon}
\newcommand{\notop}[1]{\neg(#1)}
\def\orderas{\vartheta}
\def\racefreeas{\ensuremath{\mathrm{\Psi}}}
\def\os{\mathcal{S}}
\newcommand{\osbck}[1]{{#1}^{bt}}
\newcommand{\osfrt}[1]{{#1}^{ft}}
\newcommand{\osasm}[1]{{#1}^{\oplus}}
\newcommand{\osgrd}[1]{{#1}^{\ominus}}

\newcommand{\bordrmap}[1]{{#1}.\rolemap}
\newcommand{\bordcmap}[1]{{#1}.\chanmap}

\newcommand{\racefree}[2]{\ensuremath{#1 \,{\vDash_{RF}}\,  #2}}
\newcommand{\racefreerel}[2]{{\ifthenelse{\equal{#2}{}}{RF(#1)}{RF(#1,#2)}}}
% constraint base formula
\def\cbformula{BForm}
% constraint formula
\def\cformula{EForm}
\def\eborder{\beta}
\def\tborder{\eborder^T}
\def\oborder{\eborder^E}


\definecolor{navyblue}{HTML}{336699}
\definecolor{denimblue}{HTML}{151B8D}
\newcommand\blueord[1]{\textcolor{denimblue}{#1}}
\newcommand\grayord[1]{\textcolor{gray}{#1}}

\def\fmsg{\St}
\newcommand\chanspec[3]{\chani{#1}{#2}{#3}}


\lstdefinestyle{informal}{
  belowcaptionskip=1\baselineskip,
  breaklines=true,
  % frame=L,
  % xleftmargin=\parindent,
  showstringspaces=false,
  basicstyle=\footnotesize\ttfamily\color{blue},
  keywordstyle=\bfseries\color{blue},
  commentstyle=\itshape\color{blue},
  identifierstyle=\color{blue},
  stringstyle=\color{blue},
}

%%%%%%%%%%%%%%%%%%%%%%%%%%%%%%%%%%%%%
% verification related stuff
%%%%%%%%%%%%%%%%%%%%%%%%%%%%%%%%%%%%%
\newcommand{\openall}[3]{openall(#1,#2,#3)}
\newcommand{\opened}[3]{opened(#1,#2,#3)}
\newcommand{\closed}[1]{closed(#1)}
\newcommand{\emptyc}[2]{empty(#1,#2)}
\def\simpag{\textbackslash{\sep}}
\def\init{init}
\def\initall{initall}

\newcommand{\peer}[1]{Peer(#1)}
\newcommand{\bindv}[2]{bind(#1,#2)}

%%%%%%%%%%%%%%%%%%%%%%%%%%%%%%%%%%%%%
% auxiliary functions and notations
%%%%%%%%%%%%%%%%%%%%%%%%%%%%%%%%%%%%%

\newcommand{\EV}[1]{EV(#1)}
\newcommand{\TR}[1]{TR(#1)}
\newcommand{\first}[1]{TR^{fst}(#1)}
\newcommand{\sub}[1]{sub(#1)}

\newcommand{\safesnd}[4]{\safesend(#2, #4, #3, #1)}
\newcommand{\safercv}[2]{\saferecv(#2, #1)}
\newcommand{\safefid}[3]{\fid(#1, #2, #3)}

\def\sendfunction{send}
\def\recvfunction{recv}
\def\notifyAllfunction{notifyAll}
\def\waitfunction{wait}
\newcommand{\tsend}[1]{\sendfunction(#1)}
\newcommand{\notifyAll}[1]{\notifyAllfunction(#1)}
\newcommand{\mwait}[1]{\waitfunction(#1)}
\newcommand{\msend}[2]{\sendfunction(#1, #2)}
\newcommand{\trecv}[1]{\recvfunction(#1)}
\newcommand{\esend}[1]{{\sendfunction_e}(#1)}
\newcommand{\erecv}[1]{{\recvfunction_e}(#1)}
\newcommand{\tlbl}[1]{lbl(#1)}
\newcommand{\tchan}[1]{chan(#1)}
\newcommand{\tmsg}[1]{msg(#1)}
\newcommand{\mdisj}[2]{#1 \# #2}
\newcommand{\evpeer}[1]{#1.peer}
\newcommand{\evlbl}[1]{#1.id}
\newcommand{\cnil}{\btt{NULL}}
\newcommand{\cpar}{\|} %parallel composition

\newcommand{\RF}{RF~}
\newcommand{\SO}{SO}
\newcommand{\ltseq}[2]{#1\!{\prec}#2}
\newcommand{\ord}[1]{{\ifthenelse{\equal{#1}{}}{seq}{seq(#1)}}}
\newcommand{\adj}[2]{Adj(#1,#2)}
\newcommand{\link}[2]{Adj^+(#1,#2)}
\def\proof{\vspace{-0.2cm}\noindent{\it Proof}\hspace*{0.5cm}}
\renewenvironment{proof}{\noindent{\it Proof:}}{\qedsymbol}
\newcommand{\satisf}[3]{\ensuremath{(#1,#2) \vDash #3}}
\def\smstack{\myit{s}}
\def\smheap{\myit{h}}
\newcommand{\node}[3]{\hformn{#1}{#2}{#3}}
\def\disjoint{\perp}
\def\mord{\myit{o}}
\newcommand{\eeval}[1]{[[#1]]}
\newcommand{\beval}[1]{[[#1]]}


\begin{abstract}
Distributed software is ubiquitous in today’s digital world. The expert designers of distributed systems often engage formal techniques to provide an unambiguous description of such software. One rigorous trend in describing the communication/synchronization of distributed applications is that of multiparty session protocols. A multiparty session protocol describes the communication interaction between multiple peers, highlighting the order and the kind of transmitted messages.

Coming up with the right specification is often challenging due to the rather complex interaction schemes used within these systems. To ease the task of designing and writing session protocols and to test their feasibility before even implementing them, it would be appropriate to be able to visualize and simulate the intended communication. "One Look Is Worth A Thousand Words" they say! The goal is to have a visualization mechanism of session protocols that goes two ways. On one hand it generates a graphical display of the communication given the formal protocol. On the other hand, it should be able to allow the designer to interact with the graphical protocol such that any modification in the graphic protocol would also be captured by its formal counterpart.

\end{abstract}

\maketitle


\lstdefinestyle{MCStyle}{
  language=[LaTeX]Tex,
  keywordstyle=\bfseries\color[rgb]{0,0,0},
  texcsstyle=*\color{black}
  captionpos=b,
  numbers=left,
  basicstyle=\scriptsize\normalfont,
  stepnumber=1,
  numbersep=10pt,
  tabsize=2,
  showspaces=false,
  showstringspaces=false,
  numbersep=0pt,
}

\lstdefinelanguage{MC}
{
	morekeywords={data,int},
	sensitive=false,
	morecomment=[l]{@},
	morecomment=[s]{/*}{*/},
	morestring=[b]"
}



\section{Overview}

\begin{center}
\includegraphics[scale=0.6]{overview.png}
\end{center}

In this setting a user should be able to write a global protocol with
any of the following synchronized mechanisms:
(1) formal language in the style of session logic \cite{Andreea2017thesis},
(2) the DSL for visualizing a communication protocol,
(3) the GUI, which should provide a drag-and-drop facility for adding new
communicating parties and or transmissions.

Any change in one of these three layers should be uniformly reflected in
all the layers (either automatically or via a synchronization mechanism).

\subsection{Formal Global Protocols}
\begin{center}
$
\begin{array}{llll}
  \textit{Single ~ transmission}~ & \code{\terminal}& {~ ::=}& \quad \code{\atransmit{S}{R}{v\,{\cdot}\,\fmsg}{\chanvar}{}}\\
  \textit{Global~protocol}& \code{\prot% (\rolevar^*)
  } &{~ ::=} &\code{~~~ \terminal}\\
  \textit{Concurrency} &&& ~|~ \code{\prot\,{\useq}\,\prot} \\
  \textit{Choice} &&& ~|~ \code{\prot\,{\vee}\,\prot } \\
  \textit{Sequencing}
  &&& ~|~ \code{\prot\,{\gseq{}}\,\prot} \\
  % \textit{Guard} &&& ~|~ \code{\guard{\racefreeas}} \\
  % \textit{Assumption} &&& ~|~ \code{\assume{\racefreeas}} \\
  \textit{Inaction} &&&   ~|~ {\emp}
\end{array}
$
\end{center}


\subsection{DSL for Building and Simulating Session Protocols}

\begin{itemize}
  \item should follow that of the Formal Global Protocols as much as
    possible.
  \item split between {\it the static} part for building the visualization
    of the protocol, and {\it the dynamic} part for simulating the
    communication described by the protocol.
\end{itemize}


\noindent
\emph{\bf Static}.
\begin{verbatim}
  {program}       Prog          :== Parties_decl x Channels_decl x Transm
  {parties decl}  Parties_decl  :== Party P*
  {channels decl} Channels_decl :== Channel c*
  {transmissions} Transm        :==  P -(OPT color)-> P: c<msg(v)>
                                   | {Transm} * {Transm}
                                   | Transm ; Transm
                                   | {Transm} \/ {Transm}
                                   | loop(Transm)
                                   | end.
\end{verbatim}

\emph{\bf Dynamic}.

\asay{Should we employ some sort of FRP here? for the simulation part.
  Fro example, Haskell's reactive-banana
  https://wiki.haskell.org/Reactive-banana/Examples
  }

\section{Session Protocols GUI}

\asay{ see examples of interacting with sequenced diagrams
  https://www.websequencediagrams.com }

we further need:
\begin{itemize}
  \item a way to represent arbitrary order / concurrency inter-party
  \item a way to represent arbitrary order / concurrency intra-party
\end{itemize}


% The bibliography should be embedded for final submission.

%\begin{thebibliography}{}
%\softraggedright
%\end{thebibliography}

%\begingroup
%\softraggedright

\bibliography{ref}
\bibliographystyle{ACM-Reference-Format}
\citestyle{acmauthoryear}

%\bibliographystyle{plain}
%\endgroup

\end{document}
